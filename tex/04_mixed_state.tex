\documentclass[10pt]{beamer}
%\usepackage{xspace}
\usepackage{amsmath,amssymb}
\usepackage{graphicx}
%\usepackage{svg}
%\usepackage{pgfpages}
%\pgfpagesuselayout{4 on 1}[a4paper,border shrink=5mm,landscape]
%\usepackage{psfrag}
%\usepackage[usenames,dvipsnames]{xcolor}
\usepackage{blkarray}
\usepackage{braket}
\usepackage{tikz}
\usepackage{tikz-3dplot}
%\usetikzlibrary{tikz-3dplot}
\usetikzlibrary{graphs}
\usetikzlibrary{datavisualization}
\usetikzlibrary{datavisualization.formats.functions}
\usepackage{pgfplotstable}
\usepgfplotslibrary{patchplots}

\setbeamercovered{transparent}

\usetheme{Pittsburgh}
%\usetheme{default}

\setbeamertemplate{sidebar right}{}
\setbeamertemplate{footline}[frame number]
%\usefonttheme{professionalfonts}

%\usepackage{sansmathaccent}
%\usepackage{bm}

%\usepackage{unicode-math}
%%\setmainfont[SlantedFont={Latin Modern Roman Slanted},SlantedFeatures={Color=000000},
%%  SmallCapsFont={TeX Gyre Termes},SmallCapsFeatures={Letters=SmallCaps}]{XITS}
%\setmathfont[math-style=ISO,sans-style=upright]{XITS Math}
%\setmathfont[range={\mathcal,\mathbfcal}]{Latin Modern Math}

\usepackage{sfmath}

%\mathversion{sans}

\newcommand{\Tr}{\mathsf{Tr}}

\definecolor{redorange}{rgb}{1.0, .25, .25}
\definecolor{citation}{rgb}{.1, 0.8, .35}
\newcommand\emm[1]{\textcolor{redorange}{{#1}}}
\newcommand\numc[1]{\textcolor{citation}{{\bf #1}}}

%\newcommand\bm[1]{{\mbox{\boldmath $#1$}}}
\newcommand\bm[1]{{\mathbf{#1}}}
%\newcommand\bm[1]{{\bf #1}}
%\newcommand\bm[1]{\ensuremath{\boldsymbol{#1}}}
%\newcommand\bm[1]{{\textbf{\it #1}}}

\title{Spectral decomposition, purification and superdense coding}
\author{Ryuhei Mori}
%\institute{$\vcenter{\hbox{\includegraphics[width=30pt]{ELC_logo}}}$ Postdoctoral Fellow of ELC\\ $\vcenter{\hbox{\includegraphics[width=20pt]{titech_logo}}}$ Tokyo Institute of Technology}
\institute{Tokyo Institute of Technology}
\date{12, Oct., 2021}



\begin{document}
\begin{frame}[plain]
\maketitle
\end{frame}

\begin{frame}{Pauli matrices in braket notation}
\begin{itemize}
\setlength{\itemsep}{3em}
\item
\begin{equation*}
Z=\begin{bmatrix}1&0\\0&-1\end{bmatrix}=
\ket{0}\bra{0}
-\ket{1}\bra{1}
\end{equation*}
\item
$\ket{+}:=(\ket{0}+\ket{1})/\sqrt{2},\quad  \ket{-}:=(\ket{0}-\ket{1})/\sqrt{2}$.
\begin{equation*}
X=\begin{bmatrix}0&1\\1&0\end{bmatrix}=
\ket{+}\bra{+}
-\ket{-}\bra{-}
\end{equation*}
\item
$\ket{a}:=(\ket{0}+i\ket{1})/\sqrt{2},\quad  \ket{b}:=(\ket{0}-i\ket{1})/\sqrt{2}$.
\begin{equation*}
Y=\begin{bmatrix}0&-i\\i&0\end{bmatrix}=
\ket{a}\bra{a}
-\ket{b}\bra{b}
%\frac12\begin{bmatrix}1\\i\end{bmatrix}\begin{bmatrix}1&-i\end{bmatrix}
%-\frac12\begin{bmatrix}1\\-i\end{bmatrix}\begin{bmatrix}1&i\end{bmatrix}
\end{equation*}
\end{itemize}
\end{frame}

\begin{frame}{Spectral decomposition theorem}
\begin{definition}[Normal operator]
$A\in\mathcal{L}(\mathcal{X})$ is said to be \emm{normal} if $AA^\dagger = A^\dagger A$.
\end{definition}

\vspace{1em}
Hermitian matrix ($H^\dagger = H$) and unitary matrix ($UU^\dagger = I$) are normal.

\vspace{2em}
\begin{theorem}[Spectral decomposition theorem]
$A\in \mathcal{L}(\mathbb{C}^n)$ is \emm{normal} if and only if there exist orthonormal basis $\{\ket{\psi_j}\}$ of $\mathbb{C}^n$ and complex numbers $\{\lambda_j\}$ such that
\begin{equation*}
A=\sum_j \lambda_j\ket{\psi_j}\bra{\psi_j}.
\end{equation*}
\end{theorem}
\end{frame}

\if0
\begin{frame}{Spectral decomposition of Hermitian operator}
\begin{theorem}[Spectral decomposition theorem]
$H\in \mathcal{L}(\mathbb{C}^n)$ is \emm{Hermitian} if and only if there exist orthonormal basis $\{\ket{\psi_i}\}$ of $\mathbb{C}^n$ and real numbers $\{\lambda_i\}$ such that
\begin{equation*}
H=\sum_i \lambda_i\ket{\psi_i}\bra{\psi_i}.
\end{equation*}
\end{theorem}
\begin{proof}[Outline of the proof]
\begin{itemize}
\item Any complex matrix $H$ has an eigenvalue and eigenvector.
\item All eigenvalues of any Hermitian matrix $H$ are real.
\item Any Hermitian matrix $H$ has a spectral decomposition.
\end{itemize}
\end{proof}
\end{frame}
\fi

\begin{frame}{Any complex matrix has an eigenvalue}
For any $A\in \mathcal{L}(\mathbb{C}^n)$ and non-zero $\ket{\psi}\in \mathbb{C}^n$,
\begin{equation*}
\ket{\psi},\, A\ket{\psi},\, A^2\ket{\psi},\,\dotsc,A^n\ket{\psi}
\end{equation*}
are linearly \emm{dependent}.
There exist $a_0,\dotsc,a_n$ that are not all-zero satisfying
\begin{align*}
0 &= a_0\ket{\psi}+a_1 A\ket{\psi}+\dotsb+a_nA^n\ket{\psi}\\
 &= a_m(A-\lambda_1 I)(A-\lambda_2 I)\dotsm(A-\lambda_m I)\ket{\psi}
\end{align*}
where $m$ is the largest $i$ such that $a_i\ne 0$.

%This means at least one $A-\lambda_i I$ is not injective, i.e.,
This means that there exist $i\in\{1,2,\dotsc, m\}$ and non-zero $\ket{\varphi}\in\mathbb{C}^n$ such that
\begin{align*}
(A-\lambda_i I)\ket{\varphi} = 0.
%\iff
%Aw = \lambda_i w.
\end{align*}
\end{frame}

\if0
\begin{frame}{All eigenvalues of Hermitian matrix are real}
Assume that Hermitian matrix $H$ has an eigenvalue $\lambda$ and a corresponding eigenvector $v$.
\begin{equation*}
\lambda \langle v, v\rangle
=\langle \lambda v, v\rangle
=\langle H v, v\rangle
\,\emm{=}\,\langle v, H v\rangle
=\langle v, \lambda v\rangle
=\lambda^* \langle v, v\rangle.
\end{equation*}
Hence, $\lambda$ is real.
%\begin{equation*}
%\lambda \langle v, w\rangle
%=\langle \lambda v, w\rangle
%\end{equation*}

\end{frame}
\fi

\begin{frame}{Orthogonal projection}
For linear space $V$ and its subspace $W$,
the \emm{orthogonal projection} onto $W$ is defined by
\begin{align*}
P = \sum_{j} \ket{\psi_j}\bra{\psi_j}
\end{align*}
where $(\ket{\psi_j})_j$ forms an orthonormal basis of $W$.

\vspace{2em}
\begin{itemize}
\item $P$ is Hermitian
\item $P^2 = P$
\item $P\ket{\psi}\in W$ for any $\ket{\psi}\in V$
\item $P\ket{\psi}=\ket{\psi}$ for any $\ket{\psi}\in W$
\item $I-P$ is the orthogonal projection onto $W_\perp$
\end{itemize}
\end{frame}

\begin{frame}{Any normal matrix has a spectral decomposition}
\emm{Induction on the dimension} $n$. Spectral decomposition theorem obviously holds for $n=1$.
$A$ has a eigenvalue $\lambda$ and corresponding eigenspace $W$.
Let $P$ be the orthogonal projection onto $W$.
Let $Q=I-P$.
\begin{align*}
A=(P+Q)A(P+Q)&=PAP+PAQ+QAP+QAQ
\end{align*}
\begin{itemize}
\item $PAP=\lambda P$.
\item $QAP=Q\lambda P=0$.
\item For $\ket{\psi}\in W$, $A A^\dagger\ket{\psi} = A^\dagger A\ket{\psi} = \lambda A^\dagger\ket{\psi}$ which means $A^\dagger\ket{\psi}\in W$.
This implies $(PAQ)^\dagger = QA^\dagger P = 0$.
\end{itemize}

\vspace{1em}
Hence, $A=\lambda P + QAQ$.
Since $QA=QA(P+Q)=QAQ$ and $QA^\dagger = QA^\dagger(P+Q) = QA^\dagger Q$,
\begin{align*}
&(QAQ)(QA^\dagger Q) = QA A^\dagger Q\\
&= QA^\dagger AQ = QA^\dagger QQ AQ
\end{align*}
Hence, $QAQ$ is normal linear operator on $W_\perp$.
From the hypothesis of induction, $QAQ$ has a spectral decomposition.
\end{frame}

\begin{frame}{Terminology}
\begin{itemize}
\setlength{\itemsep}{1.5em}
\item \emm{Density matrix}, density operator: A Hermitian matrix $\rho$ that represents a state, i.e., $\rho\succeq 0$, $\Tr(\rho)=1$.
\item \emm{Pure state}: A state that cannot be written as a convex combination of other states.
Equivalently, its a density operator with rank one.
\item \emm{State vector}: A complex unit vector $\ket{\psi}$ that represents a pure state $\rho=\ket{\psi}\bra{\psi}$.
\item \emm{Mixed state}: A state that is not a pure state.
\item Positive operator-valued measurement (\emm{POVM}): A tuple  $\{P_j\}$ of Hermitian matrices that represents a measurement, i.e.,
$P_j\succeq0$ and $\sum_jP_j=I$.
\end{itemize}
\end{frame}

\begin{frame}{Ensemble of states}
Let $\rho_1,\dots,\rho_k$ be density matrices.
If $\rho_i$ is prepared with probability $p_i$, and POVM $\{P_j\}$ is applied,
outcome $j$ is obtained with probability
\begin{equation*}
\sum_{i=1}^k p_i \Tr(\rho_i P_j)
=\Tr\left(\emm{\sum_{i=1}^k p_i \rho_i} P_j\right).
\end{equation*}

\vspace{2em}
Hence, this ensemble of states is represented by $\rho:=\emm{\sum_i p_i\rho_i}$.
\end{frame}

\begin{frame}{Ensemble of pure states}
Any quantum state
\begin{equation*}
\rho=\sum_i \lambda_i\ket{\psi_i}\bra{\psi_i}
\end{equation*}
for $(\lambda_i\ge 0)_i$
can be regarded as an \emm{ensemble} $(\lambda_i,\ket{\psi_i}\bra{\psi_i})_i$ of pure states.

\vspace{1em}
\begin{align*}
\rho&=\frac34\ket{0}\bra{0} + \frac14\ket{1}\bra{1}\\
&=\frac12\ket{a}\bra{a} + \frac12\ket{b}\bra{b}
\end{align*}
for
\begin{align*}
\ket{a}&:=\sqrt{\frac34}\ket{0}+\sqrt{\frac14}\ket{1}\\
\ket{b}&:=\sqrt{\frac34}\ket{0}-\sqrt{\frac14}\ket{1}.
\end{align*}
\end{frame}

\begin{frame}{Observable}
Let $\{P_j\}$ be a POVM.
If we assign real value $a_j$ for each outcome $j$, its expectation is
\begin{align*}
\mathbb{E}[a]:=
\sum_j a_j \Tr(\rho P_j)
=
\Tr\left(\rho \,\emm{\sum_j a_j P_j}\right)
=\Tr(\rho A).
\end{align*}
Here, Hermitian operator $A:=\emm{\sum_j a_jP_j}$ is called a \emm{observable}.

\vspace{1em}
If $\{P_j\}$ is a \emm{projective measurement}, i.e., $P_jP_k = \delta_{j,k}P_j$,
\begin{align*}
\mathbb{E}[a^n]:=
\sum_j a_j^n \Tr(\rho P_j)
=
\Tr\left(\rho \sum_j a_j^n P_j\right)
=\Tr(\rho A^n).
\end{align*}
For example, $X$ and $Z$ are observables for POVMs $\{\ket{+}\bra{+},\ket{-}\bra{-}\}$ and $\{\ket{0}\bra{0},\ket{1}\bra{1}\}$ with the assignments $\pm1$, respectively.
\end{frame}

%\begin{frame}{No-cloning theorem}
%\end{frame}

\begin{frame}{Decoherence}
For orthonormal basis $\{\ket{\psi_i}\}$, POVM $\{\ket{\psi_i}\bra{\psi_i}\}$ is performed to a quantum state $\rho$.
If outcome is $i$, the quantum state $\rho$ is \emm{transformed} into $\ket{\psi_i}\bra{\psi_i}$.
If we \emm{don't see} the measurement outcome, the state after the measurement is
\begin{align*}
\sum_i \Tr(\rho \ket{\psi_i}\bra{\psi_i})\ket{\psi_i}\bra{\psi_i}
=
\sum_i \bra{\psi_i}\rho\ket{\psi_i}\ket{\psi_i}\bra{\psi_i}
\end{align*}
\begin{align*}
\rho=\sum_{i,j}\rho_{i,j}\ket{\psi_i}\bra{\psi_j}
\longmapsto
\sum_{i}\rho_{i,i}\ket{\psi_i}\bra{\psi_i}
\end{align*}

\vspace{2em}
This phenomenon is called \emm{decoherence}.
\end{frame}

\if0
\begin{frame}{Singular decomposition}
\end{frame}

\begin{frame}{Schmidt decomposition}
\begin{theorem}[Schmidt decomposition]
For any pure state $\ket{\psi}\in V\otimes W$, there exist orthonormal basis $\{\ket{v_i}\}$ of $V$ and $\{\ket{w_i}\}$ of $W$ such that
\begin{equation*}
\ket{\psi}=\sum_i\lambda_i \ket{v_i}\ket{w_i}.
\end{equation*}
\end{theorem}
\begin{proof}[Sketch of a proof]
By a natural isomorphism
\begin{align*}
\Phi:& V\otimes W \to L(W,V)\\
%& v\otimes w \mapsto (u\mapsto \langle w, u\rangle v)
&\ket{v}\ket{w}\mapsto \ket{v}\bra{w}
\end{align*}
the Schmidt decomposition for $V\otimes W$ corresponds to the singular value decomposition for $L(W, V)$.
\end{proof}
\end{frame}
\fi

\begin{frame}{Partial trace}
The \emm{partial trace} $\Tr_W: \mathcal{L}(V\otimes W)\to \mathcal{L}(V)$
is linear map defined by
\begin{equation*}
A\otimes B\mapsto \emm{\Tr(B)} A
\end{equation*}
for all $A\in \mathcal{L}(V)$ and $B\in \mathcal{L}(W)$.

\vspace{2em}
For $A\in\mathcal{L}(V\otimes W)$, $A$ is a linear combination of the orthonormal basis $(\ket{a}\bra{b}\otimes\ket{c}\bra{d})_{a,b,c,d}$
\begin{align*}
A = \sum_{a,b,c,d} A_{a,b,c,d} \ket{a}\bra{b}\otimes\ket{c}\bra{d}
\end{align*}

\begin{align*}
\Tr_W(A) &= \sum_{a,b,c,d} A_{a,b,c,d} \ket{a}\bra{b}\Tr\left(\ket{c}\bra{d}\right)\\
&= \sum_{a,b,c} A_{a,b,c,c} \ket{a}\bra{b}
\end{align*}
\begin{align*}
\Tr(\Tr_W(A)) &= \sum_{a,c} A_{a,a,c,c} = \Tr(A)
\end{align*}
\end{frame}

\if0
\begin{frame}{Local tomography}
For measurements $\{P_a\}_a$ of quantum system on $V$
and $\{Q_b\}_b$ of quantum system on $W$,
a measurement $\{P_a\otimes Q_b\}_{a,b}$ in the composite system is said to be \emm{local}.

\vspace{3em}
\begin{align*}
P(a, b) = \Tr(\rho (P_a \otimes Q_b)).
\end{align*}
\end{frame}
\fi


\begin{frame}{Marginal distribution and reduced density matrix}
A probability of outcome of local measurement in a composite system is
\begin{align*}
P(a, b) = \Tr(\rho (P_a \otimes Q_b)).
\end{align*}
\begin{align*}
\sum_b P(a, b) &= \sum_b \Tr(\rho (P_a \otimes Q_b))\\
 &=  \Tr\left(\rho \left(P_a \otimes \sum_b Q_b\right)\right)\\
 &=  \Tr\left(\rho \left(P_a \otimes I\right)\right)\\
 &=  \Tr(\emm{\Tr_W(\rho)} P_a).
\end{align*}
Here,
$\emm{\Tr_W(\rho)}$ is called a reduced density matrix.
\end{frame}

\begin{frame}{Reduced state of a pure state is not necessarily pure}
A two-qubit pure state (called Bell state, Bell pair or EPR pair)
\begin{align*}
\ket{\Phi} := \frac1{\sqrt{2}}(\ket{00}+\ket{11}).
\end{align*}
\begin{align*}
\ket{\Phi}\bra{\Phi}&=\frac12(\ket{0}\bra{0}\otimes \emm{\ket{0}\bra{0}} + \ket{0}\bra{1}\otimes \emm{\ket{0}\bra{1}} \\
&\qquad + \ket{1}\bra{0}\otimes \emm{\ket{1}\bra{0}} + \ket{1}\bra{1}\otimes \emm{\ket{1}\bra{1}} )
\end{align*}
By taking the partial trace for the second qubit, we obtain a reduced density matrix $I/2$.
\end{frame}


%\begin{frame}{Conditional distribution}
%\end{frame}


\begin{frame}{Purification}
\begin{theorem}
For any density matrix $\rho$ on $V$, there exists a pure state $\ket{\psi}$ of a composite system on $V\otimes W$ for some $W$ such that $\Tr_W(\ket{\psi}\bra{\psi})=\rho$.
\end{theorem}
\begin{proof}
For a spectral decomposition of $\rho$
\begin{equation*}
\rho=\sum_i \lambda_i\ket{\psi_i}_V\bra{\psi_i}_V
\end{equation*}
let
\begin{equation*}
\ket{\psi}_{V\otimes W}:=\sum_i \sqrt{\lambda_i}\ket{\psi_i}_V\ket{i}_W
\end{equation*}
where $\{\ket{i}\}_i$ is an arbitrary orthonormal basis of $W$.
Then,
\begin{equation*}
\Tr_W(\ket{\psi}_{V\otimes W}\bra{\psi}_{V\otimes W}) = \rho.\qedhere
\end{equation*}
\end{proof}
$\ket{\psi}_{V\otimes W}$ is called a \emm{purification} of $\rho$.
\end{frame}


%\begin{frame}{Schr\"odinger picture and Heisenberg picture}
%\end{frame}

\begin{frame}{Quantum states discrimination}
Alice encodes her classical information $\{1,2,\dotsc,n\}$ into quantum states $\rho_1,\rho_2,\dotsc,\rho_n\in \mathcal{H}(\mathbb{C}^{\emm{m}})$, and send it to Bob.
Bob performs a POVM $\{P_1,\dotsc,P_n\}$ for estimating $i\in\{1,\dotsc,n\}$ that Alice encoded.
Assume Bob could estimate $i$  without error. Then,
\begin{equation*}
\Tr(\rho_i P_j) = \delta_{i,j}.
\end{equation*}
\begin{align*}
\Tr(\rho_i P_j) &= 
\Tr\left(\sum_k \lambda^{(i)}_k \ket{\psi^{(i)}_k}\bra{\psi^{(i)}_k} P_j\right)\\
&= \sum_k \lambda^{(i)}_k \bra{\psi^{(i)}_k} P_j\ket{\psi^{(i)}_k}
\end{align*}
%From $\sum_k\lambda_k=1$ and $\bra{\psi}P\ket{\psi}\le 1$, we can replace $\rho_i$ with $\ket{\psi^{(i)}_k}\bra{\psi^{(i)}_k}$
%for any $k$ with $\lambda_k>0$.

%\vspace{.5em}
%$\bra{\psi^{(i)}}P_j\ket{\psi^{(i)}}=\delta_{i,j}$ implies $\{\ket{\psi^{(i)}}\}$ is orthonormal. Hence $n\le \emm{m}$.
$\bra{\psi^{(i)}_k}P_j\ket{\psi^{(i)}_k}=\delta_{i,j}$ implies $\ket{\psi^{(i)}_k}$ is an eigenvector of $P_j$ with eigenvalue $\delta_{i,j}$.
Hence, $\braket{\psi^{(i)}_k|\psi^{(j)}_\ell}=\delta_{i,j}$ which implies $\emm{n\le m}$.
\end{frame}

\begin{frame}{Superdense coding}
Alice can send \emm{two} bits to Bob by sending a single qubit and using a shared Bell state.
\begin{align*}
\ket{\Phi_{00}} &= \frac1{\sqrt{2}}(\ket{0}_A\ket{0}_B+\ket{1}_A\ket{1}_B)\\
\ket{\Phi_{01}} &= \frac1{\sqrt{2}}(\ket{1}_A\ket{0}_B+\ket{0}_A\ket{1}_B),& \text{by $X$}\\
\ket{\Phi_{10}} &= \frac1{\sqrt{2}}(\ket{0}_A\ket{0}_B-\ket{1}_A\ket{1}_B),& \text{by $Z$}\\
\ket{\Phi_{11}} &= \frac1{\sqrt{2}}(\ket{1}_A\ket{0}_B-\ket{0}_A\ket{1}_B),& \text{by $XZ$}\\
\end{align*}
These are \emm{orthogonal}.
\end{frame}

%\begin{frame}{Quantum teleportation}
%\end{frame}

\begin{frame}{Assignments}
\begin{enumerate}
\setlength{\itemsep}{2em}
\item Show the reduced density matrix $\rho_V\in \mathcal{H}(V)$ of
\begin{equation*}
\rho=\sum_{i,j}\rho_{i,j}\ket{i}\bra{j}\otimes \ket{i}\bra{j}
\in \mathcal{H}(V\otimes W)
\end{equation*}
where $\{\ket{i}\}$ is a orthonormal basis of $V$ and $W$.
\item For a single-qubit density matrix $\rho=\sum_{i,j=0}^1\rho_{i,j}\ket{i}\bra{j}$,
show the density matrix of an ensemble of quantum states $\rho$ and $Z\rho Z$ chosen with probabilities 1/2.
\item Show a purification of $\rho=\frac34\ket{0}\bra{0}+\frac14\ket{1}\bra{1}$.
\end{enumerate}
\end{frame}

\end{document}
