\documentclass[10pt]{beamer}
%\usepackage{xspace}
\usepackage{amsmath,amssymb}
\usepackage{graphicx}
%\usepackage{svg}
%\usepackage{pgfpages}
%\pgfpagesuselayout{4 on 1}[a4paper,border shrink=5mm,landscape]
%\usepackage{psfrag}
%\usepackage[usenames,dvipsnames]{xcolor}
\usepackage{blkarray}
\usepackage{braket}
\usepackage{qcircuit}
\usepackage{tikz}
\usepackage{tikz-3dplot}
%\usetikzlibrary{tikz-3dplot}
\usetikzlibrary{graphs}
\usetikzlibrary{datavisualization}
\usetikzlibrary{datavisualization.formats.functions}
\usepackage{pgfplotstable}
\usepgfplotslibrary{patchplots}

\setbeamercovered{transparent}

\usetheme{Pittsburgh}
%\usetheme{default}

\setbeamertemplate{sidebar right}{}
\setbeamertemplate{footline}[frame number]
%\usefonttheme{professionalfonts}

%\usepackage{sansmathaccent}
%\usepackage{bm}

%\usepackage{unicode-math}
%%\setmainfont[SlantedFont={Latin Modern Roman Slanted},SlantedFeatures={Color=000000},
%%  SmallCapsFont={TeX Gyre Termes},SmallCapsFeatures={Letters=SmallCaps}]{XITS}
%\setmathfont[math-style=ISO,sans-style=upright]{XITS Math}
%\setmathfont[range={\mathcal,\mathbfcal}]{Latin Modern Math}

\usepackage{sfmath}

%\mathversion{sans}

\newcommand{\Tr}{\mathsf{Tr}}

\definecolor{redorange}{rgb}{1.0, .25, .25}
\definecolor{citation}{rgb}{.1, 0.8, .35}
\newcommand\emm[1]{\textcolor{redorange}{{#1}}}
\newcommand\numc[1]{\textcolor{citation}{{\bf #1}}}

%\newcommand\bm[1]{{\mbox{\boldmath $#1$}}}
\newcommand\bm[1]{{\mathbf{#1}}}
%\newcommand\bm[1]{{\bf #1}}
%\newcommand\bm[1]{\ensuremath{\boldsymbol{#1}}}
%\newcommand\bm[1]{{\textbf{\it #1}}}

\title{Quantum teleportation}
\author{Ryuhei Mori}
%\institute{$\vcenter{\hbox{\includegraphics[width=30pt]{ELC_logo}}}$ Postdoctoral Fellow of ELC\\ $\vcenter{\hbox{\includegraphics[width=20pt]{titech_logo}}}$ Tokyo Institute of Technology}
\institute{Tokyo Institute of Technology}
\date{15, Oct., 2020}



\begin{document}
\begin{frame}[plain]
\maketitle
\end{frame}

\begin{frame}{Time evolution of a system}
Time evolution of a system is represented by a map from a state to a state.

\vspace{2em}
\begin{equation*}
T: \text{The set of states} \to \text{the set of states}.
\end{equation*}

\vspace{2em}
\begin{equation*}
p T(\rho_1) + (1-p) T(\rho_2) = T(p\rho_1 + (1-p)\rho_2)
\end{equation*}
for any density matrices $\rho_1,\,\rho_2$ and $p\in[0,1]$.

\vspace{2em}
\begin{center}
$T:\mathcal{H}(V)\to\mathcal{H}(W)$ must be \emm{linear} (a proof is needed).
\end{center}
\end{frame}




\begin{frame}{Schr\"odinger picture and Heisenberg picture}
\begin{equation*}
\emm{T^\dagger}: \text{The set of binary measurements} \to \text{the set of binary measurements}.
\end{equation*}

\vspace{.5em}
\begin{equation*}
\langle T(\rho), P\rangle
=
\langle \rho, \emm{T^\dagger}(P)\rangle
\end{equation*}
for any $\rho\in \mathcal{H}(V)$ and $P\in \mathcal{H}(W)$.
$\emm{T^\dagger}$ is an \emm{adjoint} map of $T$.
\begin{align*}
&\langle T_3(T_2(T_1(\rho))), P\rangle
=
\langle T_2(T_1(\rho)), T_3^\dagger(P)\rangle\\
=&
\langle T_1(\rho), T_2^\dagger(T_3^\dagger(P))\rangle
= \langle \rho, T_1^\dagger(T_2^\dagger(T_3^\dagger(P)))\rangle
\end{align*}
\end{frame}

\begin{frame}{No-cloning theorem}
\begin{align*}
\ket{0}\bra{0} &\longmapsto \ket{0}\bra{0} \otimes \ket{0}\bra{0} \\
\ket{1}\bra{1} &\longmapsto \ket{1}\bra{1} \otimes \ket{1}\bra{1}
\end{align*}
From the linearlity,
\begin{align*}
\frac12(\ket{0}\bra{0} + \ket{1}\bra{1}) &\longmapsto \frac12(\ket{0}\bra{0} \otimes \ket{0}\bra{0} + \ket{1}\bra{1} \otimes \ket{1}\bra{1}) \\
\end{align*}
This is not equal to
\begin{align*}
\frac12(\ket{0}\bra{0} + \ket{1}\bra{1})\otimes \frac12(\ket{0}\bra{0} + \ket{1}\bra{1}).
\end{align*}
\end{frame}

\begin{frame}{Axioms for quantum channel}
$T\colon \mathcal{H}(V) \to \mathcal{H}(W)$.

\vspace{1em}
\begin{enumerate}
\setlength{\itemsep}{2em}
\item Trace-preserving: $\Tr(T(\rho)) = \Tr(\rho)$.
\onslide<1>{\item Positive : $T(\rho)\succeq 0$ for any $\rho\succeq 0$.}
\onslide<2>{\item Completely positive: $\mathsf{id}\otimes T$ is positive.}
\end{enumerate}

%\vspace{3em}
%In the following, we only consider $T\colon \rho \longmapsto U\rho U^\dagger$.
\end{frame}

\begin{frame}{Positive but not completely positive 1/2}
\small
$T\in\mathcal{L}(\mathcal{H}(\mathbb{C}^2))\colon \text{\emm{transposition}}$ according to $\{\ket{0},\ket{1}\}$.

\vspace{2em}
The transposition is obviously trace-preserving.

\vspace{1em}
The transposition is \emm{positive}.
\begin{proof}
For any $A\succeq 0$ and $\ket{\psi}\in\mathbb{C}^2$,
\begin{align*}
\bra{\psi} T(A)\ket{\psi} &=
\bra{\psi} A^T \ket{\psi} =
\bra{\psi} A^* \ket{\psi} =
(\bra{\psi}^* A \ket{\psi}^*)^*\ge 0
\qedhere
\end{align*}
\end{proof}
\end{frame}

\begin{frame}{Positive but not completely positive 2/2}
\small
But, the transposition is \emm{not} completely positive.
\begin{proof}
For $\ket{\Phi}:=\frac1{\sqrt{2}}(\ket{00}+\ket{11})$,
\begin{align*}
&(\mathsf{id}_{\mathcal{H}(\mathbb{C}^2)}\otimes T)(\ket{\Phi}\bra{\Phi})\\
&=
(\mathsf{id}_{\mathcal{H}(\mathbb{C}^2)}\otimes T)\Biggl(\frac12\biggl(\ket{0}\bra{0}\otimes\emm{\ket{0}\bra{0}}+\ket{0}\bra{1}\otimes\emm{\ket{0}\bra{1}}\\
&\qquad +\ket{1}\bra{0}\otimes\emm{\ket{1}\bra{0}}+\ket{1}\bra{1}\otimes\emm{\ket{1}\bra{1}}\biggr)\Biggr)\\
&=
\frac12\biggl(\ket{0}\bra{0}\otimes\emm{\ket{0}\bra{0}}+\ket{0}\bra{1}\otimes\emm{\ket{1}\bra{0}}\\
&\qquad +\ket{1}\bra{0}\otimes\emm{\ket{0}\bra{1}}+\ket{1}\bra{1}\otimes\emm{\ket{1}\bra{1}}\biggr)
\end{align*}

\vspace{-.5em}
\begin{align*}
\ket{00}&\mapsto\ket{00}&
\ket{01}&\mapsto\ket{10}&
\ket{10}&\mapsto\ket{01}&
\ket{11}\mapsto\ket{11}
\end{align*}
Hence, $\ket{01}-\ket{10}\mapsto \ket{10}-\ket{01}$.
$(\mathsf{id}_{\mathcal{H}(\mathbb{C}^2)}\otimes T)(\ket{\Phi}\bra{\Phi})$ is not positive semidefinite.
\end{proof}
\end{frame}

\begin{frame}{Unitary operations}
\begin{equation*}
\rho \longmapsto U\rho U^\dagger.
\end{equation*}
\begin{enumerate}
\setlength{\itemsep}{2em}
\item Trace-preserving: $\Tr(U\rho U^\dagger) = \Tr(\rho)$.
\item Completely positive: 
$(\mathsf{id}\otimes T)(\rho) = (I\otimes U)\rho (I\otimes U^\dagger)$.
%$\mathsf{id}\otimes T$ is positive.
\end{enumerate}

\vspace{2em}
\begin{center}
In the most of quantum computing, only \emm{pure} states and \emm{unitary} operations are used.
\end{center}
\end{frame}

\begin{frame}{Quantum circuit}
\[
\Qcircuit @C=1.8em @R=1.8em @!R {
\lstick{\ket{0}}& \qw      & \gate{H} & \qw & \gate{X} & \ctrl{2} & \gate{H} &\meter\qw\\
\lstick{\ket{0}}& \qw      & \gate{H} & \ctrl{1} & \gate{Z} & \qw & \gate{H} &\meter\qw\\
\lstick{\ket{0}}& \gate{X} & \gate{H} & \targ    & \qw      & \targ    & \gate{H} &\meter\qw
}
\]
\end{frame}

\begin{frame}{Controlled not}
\begin{center}
\mbox{
\Qcircuit @C=1em @R=2.0em {
  \lstick{\ket{x}} &  \ctrl{1} & \rstick{\ket{x}} \qw\\
  \lstick{\ket{y}} &  \targ    & \rstick{\ket{y\oplus x}} \qw
}
}
\end{center}
\begin{align*}
\mathsf{CNOT} \ket{x}\ket{y} \longmapsto \ket{x}\ket{y\oplus x}
\end{align*}
\begin{align*}
\mathsf{CNOT}=
\begin{bmatrix}
1&0&0&0\\
0&1&0&0\\
0&0&0&1\\
0&0&1&0
\end{bmatrix}.
\end{align*}
\end{frame}

\if0
\begin{frame}{Simplification of quantum circuit 1/5}
\[
\Qcircuit @C=1.8em @R=1.8em @!R {
& \gate{X} & \gate{X} &  \qw
}
\quad
=
\quad
\Qcircuit @C=1.8em @R=1.8em @!R {
& \qw & \qw
}
\]
\[
\Qcircuit @C=1.8em @R=1.8em @!R {
& \gate{Z} & \gate{Z} &  \qw
}
\quad
=
\quad
\Qcircuit @C=1.8em @R=1.8em @!R {
& \qw & \qw
}
\]
\[
\Qcircuit @C=1.8em @R=1.8em @!R {
& \gate{H} & \gate{H} &  \qw
}
\quad
=
\quad
\Qcircuit @C=1.8em @R=1.8em @!R {
& \qw & \qw
}
\]


\vspace{1em}
\[
\Qcircuit @C=1.8em @R=1.8em @!R {
& \gate{H} & \gate{X} &  \gate{H}& \qw
}
\quad
=
\quad
\Qcircuit @C=1.8em @R=1.8em @!R {
& \gate{Z} & \qw
}
\]
\[
\Qcircuit @C=1.8em @R=1.8em @!R {
& \gate{H} & \gate{Z} &  \gate{H}& \qw
}
\quad
=
\quad
\Qcircuit @C=1.8em @R=1.8em @!R {
& \gate{X} & \qw
}
\]

\vspace{1em}
\[
\Qcircuit @C=1.8em @R=1.8em @!R {
& \gate{H} & \gate{X} & \qw
}
\quad
=
\quad
\Qcircuit @C=1.8em @R=1.8em @!R {
& \gate{Z} & \gate{H} & \qw
}
\]
\[
\Qcircuit @C=1.8em @R=1.8em @!R {
& \gate{H} & \gate{Z} & \qw
}
\quad
=
\quad
\Qcircuit @C=1.8em @R=1.8em @!R {
& \gate{X} & \gate{H} & \qw
}
\]
\end{frame}

\begin{frame}{Simplification of quantum circuit 2/5}
\[
\Qcircuit @C=1.8em @R=1.8em @!R {
& \gate{X} & \gate{Z} & \qw
}
\quad
=
\quad
\Qcircuit @C=1.8em @R=1.8em @!R {
& \gate{Z} & \gate{X} & \qw
}
\]
\end{frame}

\begin{frame}{Simplification of quantum circuit 3/5}
\[
\Qcircuit @C=1.8em @R=1.8em @!R {
& \qw & \ctrl{1} & \qw\\
& \gate{X} & \targ & \qw
}
\quad
=
\quad
\Qcircuit @C=1.8em @R=1.8em @!R {
& \ctrl{1} & \qw & \qw\\
& \targ &  \gate{X} &\qw
}
\]

\vspace{1em}
\[
\Qcircuit @C=1.8em @R=1.8em @!R {
& \gate{Z} & \ctrl{1} & \qw\\
& \qw & \targ & \qw
}
\quad
=
\quad
\Qcircuit @C=1.8em @R=1.8em @!R {
& \ctrl{1} & \gate{Z} & \qw\\
& \targ &  \qw &\qw
}
\]

\vspace{1em}
\[
\Qcircuit @C=1.8em @R=1.8em @!R {
& \gate{Z} & \ctrl{1} & \qw\\
& \qw & \ctrl{0} & \qw
}
\quad
=
\quad
\Qcircuit @C=1.8em @R=1.8em @!R {
& \ctrl{1} & \gate{Z} & \qw\\
& \ctrl{0} &  \qw &\qw
}
\]
\end{frame}

\begin{frame}{Simplification of quantum circuit 4/5}
\[
\Qcircuit @C=1.8em @R=1.8em @!R {
& \qw & \ctrl{1} & \qw\\
& \gate{H} & \targ & \qw
}
\quad
=
\quad
\Qcircuit @C=1.8em @R=1.8em @!R {
& \ctrl{1} & \qw & \qw\\
& \ctrl{0} &  \gate{H} &\qw
}
\]

\vspace{1em}
\[
\Qcircuit @C=1.8em @R=1.8em @!R {
& \qw & \ctrl{1} & \qw\\
& \gate{H} & \ctrl{0} & \qw
}
\quad
=
\quad
\Qcircuit @C=1.8em @R=1.8em @!R {
& \ctrl{1} & \qw & \qw\\
& \targ &  \gate{H} &\qw
}
\]
\end{frame}

\begin{frame}{Simplification of quantum circuit 5/5}
\[
\Qcircuit @C=1.8em @R=1.8em @!R {
& \qw & \ctrl{1} & \qw\\
& \gate{Z} & \targ & \qw
}
\quad
=
\quad
\Qcircuit @C=1.8em @R=1.8em @!R {
& \ctrl{1} & \gate{Z} & \qw\\
& \targ &  \gate{Z} &\qw
}
\]

\vspace{1em}
\[
\Qcircuit @C=1.8em @R=1.8em @!R {
& \gate{X} & \ctrl{1} & \qw\\
& \qw & \targ & \qw
}
\quad
=
\quad
\Qcircuit @C=1.8em @R=1.8em @!R {
& \ctrl{1} & \gate{X} & \qw\\
& \targ &  \gate{X} &\qw
}
\]

\vspace{1em}
\[
\Qcircuit @C=1.8em @R=1.8em @!R {
& \qw & \ctrl{1} & \qw\\
& \gate{X} & \ctrl{0} & \qw
}
\quad
=
\quad
\Qcircuit @C=1.8em @R=1.8em @!R {
& \ctrl{1} & \gate{Z} & \qw\\
& \ctrl{0} &  \gate{X} &\qw
}
\]

\end{frame}
\fi




\begin{frame}{Bell states and quantum circuit}

\vspace{1em}
\centering
\mbox{
\Qcircuit @C=1em @R=.7em {
  \lstick{\ket{0}} & \gate{H} & \ctrl{1} &  \qw & \qw \\
  \lstick{\ket{0}} & \qw      & \targ    &  \qw & \qw 
}
}

\vspace{2em}
\begin{align*}
\ket{0}\ket{0} &\longmapsto
\frac1{\sqrt{2}}(\ket{0}+\ket{1})\ket{0}
=
\frac1{\sqrt{2}}(\ket{0}\ket{0}+\ket{1}\ket{0})\\
 &\longmapsto
\frac1{\sqrt{2}}(\ket{0}\ket{0}+\ket{1}\ket{1})
\end{align*}

\vspace{1em}
\begin{align*}
\ket{x}\ket{y} &\longmapsto
\frac1{\sqrt{2}}(\ket{0}+(-1)^x\ket{1})\ket{y}
=
\frac1{\sqrt{2}}(\ket{0}\ket{y}+(-1)^x\ket{1}\ket{y})\\
 &\longmapsto
\frac1{\sqrt{2}}(\ket{0}\ket{y}+(-1)^x\ket{1}\ket{\bar{y}}).
\end{align*}

\end{frame}

\begin{frame}{Conditional density operator}
A probability of outcome of local measurement in a joint system is
\begin{align*}
P(a, b) = \Tr(\rho_{V\otimes W} (P_a \otimes Q_b)).
\end{align*}
\begin{align*}
P(a\mid b) &= \frac1{P(b)} \Tr(\rho_{V\otimes W} (P_a \otimes Q_b))
=  \frac1{P(b)}\Tr(\emm{\Tr_W(\rho_{V\otimes W} (I_V\otimes Q_b))} P_a).
\end{align*}
$\rho_{V\mid Q_b} := \emm{\frac1{P(b)} {\Tr_W(\rho_{V\otimes W} (I_V\otimes Q_b))}}$.

\vspace{2em}
For $\rho_{V\otimes W}=\ket{\varphi}_{V\otimes W}\bra{\varphi}_{V\otimes W}$ and $Q_b=\ket{\psi_b}_W\bra{\psi_b}_W$,
\begin{equation*}
\emm{\Tr_W(\rho_{V\otimes W} (I\otimes Q_b))}
=
\Tr_W(\ket{\varphi}_{V\otimes W}\bra{\varphi}_{V\otimes W} (I_V\otimes \ket{\psi_b}_W\bra{\psi_b}_W))
\end{equation*}
\end{frame}

\begin{frame}{Conditional density operator for pure state}
\small
For $\rho=\ket{\varphi}_{V\otimes W}\bra{\varphi}_{V\otimes W}$ and $Q_b=\ket{\psi_b}_W\bra{\psi_b}_W$,
\begin{equation*}
\Tr_W(\rho_{V\otimes W} (I_V\otimes Q_b))
=
\Tr_W(\ket{\varphi}_{V\otimes W}\bra{\varphi}_{V\otimes W} (I\otimes \ket{\psi_b}_W\bra{\psi_b}_W))
\end{equation*}
From an expression $\ket{\varphi}_{V\otimes W}=\emm{\sum_{i,j}\varphi_{i,j} \ket{i}_V\ket{\psi_j}_W}$,
\begin{align*}
&\Tr_W(\ket{\varphi}_{V\otimes W}\bra{\varphi}_{V\otimes W} (I_V\otimes \ket{\psi_b}_W\bra{\psi_b}_W))\\
&=
\Tr_W\left(\sum_{i,j,k,l} \varphi_{i,j}\varphi^*_{k,l} \ket{i}_V\ket{\psi_j}_W\bra{k}_V\bra{\psi_l}_W (I_V\otimes \ket{\psi_b}_W\bra{\psi_b}_W)\right)\\
&=\Tr_W\left(\sum_{i,j,k,l} \varphi_{i,j}\varphi^*_{k,l} \ket{i}_V\bra{k}_V\otimes \ket{\psi_j}_W\bra{\psi_l}_W (I_V\otimes \ket{\psi_b}_W\bra{\psi_b}_W)\right)\\
&=\sum_{i,j,k,l} \varphi_{i,j}\varphi^*_{k,l} \ket{i}_V\bra{k}_V \Tr\left(\ket{\psi_j}_W\bra{\psi_l}_W \ket{\psi_b}_W\bra{\psi_b}_W\right)\\
&=\sum_{i,k} \varphi_{i,b}\varphi^*_{k,b} \ket{i}_V\bra{k}_V
=
\left(\sum_{i} \varphi_{i,b}\ket{i}_V\right)\left(\sum_{k} \varphi^*_{k,b} \bra{k}_V\right)
\end{align*}
\begin{align*}
\ket{\varphi}_{V\otimes W}=\sum_{i,j}\varphi_{i,j} \ket{i}_V\ket{\psi_j}_W
\longmapsto
\emm{\frac1{\sqrt{P(b)}}\sum_{i} \varphi_{i,b}\ket{i}_V}
\end{align*}
\end{frame}

\begin{frame}{Examples of conditional density operator}
For $\ket{\psi}_{V\otimes W} := \sum_{i,j=0}^1 \alpha_{i,j}\ket{i}_V\ket{j}_W$, we measure the system $W$ by $(\ket{0}\bra{0},\ket{1}\bra{1})$.

\vspace{2em}
if the outcome is 0, the state $\frac1{\sqrt{|\alpha_{0,0}|^2+|\alpha_{1,0}|^2}}\sum_{i=0}^1 \alpha_{i,0}\ket{i}_V$.

\vspace{2em}
if the outcome is 1, the state $\frac1{\sqrt{|\alpha_{0,1}|^2+|\alpha_{1,1}|^2}}\sum_{i=0}^1 \alpha_{i,1}\ket{i}_V$.
\end{frame}


\begin{frame}{Quantum teleportation}

\[
\Qcircuit @C=1.8em @R=0.8em @!R {
\lstick{\ket{\psi}_{A_1}}& \qw & \qw & \ctrl{1} & \gate{H} & \meter & \cctrl{2}\\
\lstick{\ket{0}_{A_2}} & \qw     & \targ & \targ & \qw & \meter \cwx[1] &\\
\lstick{\ket{0}_B} & \gate{H}     & \ctrl{-1}   & \qw & \qw   & \gate{X} & \gate{Z} & \rstick{\ket{\psi}} \qw
}
\]
\end{frame}

\begin{frame}{Quantum teleportation}
\small
\begin{align*}
\ket{\Phi}_{A_2\otimes B} &= \frac1{\sqrt{2}}\left(\ket{0}_{A_2}\ket{0}_B + \ket{1}_{A_2}\ket{1}_B\right)
\end{align*}
\begin{align*}
(H_{A_1}\otimes I_{A_2}\otimes I_B)(\mathsf{CNOT}_{A_1A_2}\otimes I_B)\ket{\psi}_{A_1}\ket{\Phi}_{A_2\otimes B}
\end{align*}
For $\ket{\psi}_{A_1}=\alpha\ket{0}_{A_1}+\beta\ket{1}_{A_1}$
\begin{align*}
\ket{\psi}_{A_1}\ket{\Phi}_{A_2\otimes B} &= \left(\alpha\ket{0}_{A_1}+\beta\ket{1}_{A_1}\right)\frac1{\sqrt{2}}\left(\ket{0}_{A_2}\ket{0}_B + \ket{1}_{A_2}\ket{1}_B\right)\\
 &\overset{\emm{\mathsf{CNOT}_{A_1A_2}}}{\longmapsto} \frac{\alpha}{\sqrt{2}}\left(\ket{0}_{A_1}\ket{0}_{A_2}\ket{0}_B + \ket{0}_{A_1}\ket{1}_{A_2}\ket{1}_B\right)\\
 &\qquad + \frac{\beta}{\sqrt{2}}\left(\ket{1}_{A_1}\ket{1}_{A_2}\ket{0}_B + \ket{1}_{A_1}\ket{0}_{A_2}\ket{1}_B\right)\\
 &\overset{\emm{H_{A_1}}}{\longmapsto} \frac{\alpha}{2}\left(\ket{000}_{A_1A_2B} +\ket{100}_{A_1A_2B} + \ket{011}_{A_1A_2B} + \ket{111}_{A_1A_2B}\right)\\
 &\quad + \frac{\beta}{2}\left(\ket{010}_{A_1A_2B} -\ket{110}_{A_1A_2B} + \ket{001}_{A_1A_2B} - \ket{101}_{A_1A_2B}\right)
\end{align*}
\end{frame}

\begin{frame}{Quantum teleportation}
\begin{align*}
 &\frac{\alpha}{2}\left(\ket{000}_{A_1A_2B} +\ket{100}_{A_1A_2B} + \ket{011}_{A_1A_2B} + \ket{111}_{A_1A_2B}\right)\\
 &\quad + \frac{\beta}{2}\left(\ket{010}_{A_1A_2B} -\ket{110}_{A_1A_2B} + \ket{001}_{A_1A_2B} - \ket{101}_{A_1A_2B}\right)\\
&=\frac12\Biggl[\emm{\ket{00}_{A_1A_2}}\left(\alpha\ket{0}_B+\beta\ket{1}_B\right) + \emm{\ket{01}_{A_1A_2}}\left(\alpha\ket{1}_B+\beta\ket{0}_B\right)\\
&\qquad + \emm{\ket{10}_{A_1A_2}}\left(\alpha\ket{0}_B-\beta\ket{1}_B\right) + \emm{\ket{11}_{A_1A_2}}\left(\alpha\ket{1}_B-\beta\ket{0}_B\right)\Biggr]
\end{align*}

According to the measurement outcome of $A_1A_2$
\begin{align*}
00 &\Rightarrow I\\
01 &\Rightarrow X\\
10 &\Rightarrow Z\\
11 &\Rightarrow ZX\\
\end{align*}
\end{frame}

\if0
\begin{frame}{Transpose trick}
\small
\begin{align*}
\ket{\Phi} := \frac1{\sqrt{d}} \sum_{j=0}^{d-1} \ket{j}\ket{j}
\end{align*}
\begin{lemma}
\begin{align*}
(A\otimes I) \ket{\Phi} &= (I\otimes A^T)\ket{\Phi}
\end{align*}
\end{lemma}
\begin{proof}
\begin{align*}
&\mathcal{M}\left((A\otimes I) \ket{\Phi}\right)
=\mathcal{M}\left(\frac1{\sqrt{d}}\sum_{j=0}^{d-1}A\ket{j}\ket{j}\right)\\
&=\frac1{\sqrt{d}}\sum_{j=0}^{d-1}A\ket{j}\bra{j}
=\frac1{\sqrt{d}}A \sum_{j=0}^{d-1}\ket{j}\bra{j}
=\frac1{\sqrt{d}}\sum_{j=0}^{d-1}\ket{j}\bra{j}A\\
&\mathcal{M}\left((I\otimes A^T) \ket{\Phi}\right)
=\frac1{\sqrt{d}}\sum_{j=0}^{d-1}\ket{j}(A^T\ket{j})^T\\
\end{align*}
\end{proof}
\end{frame}

\begin{frame}{BB84 protocol}
\end{frame}
\fi

\begin{frame}{Assignments}
\begin{enumerate}
\setlength{\itemsep}{2em}
\item Show the state vector $\ket{\psi}\in \mathbb{C}^2$ of the Bell state
\begin{equation*}
\frac1{\sqrt{2}}(\ket{00}+\ket{11}) \in \mathbb{C}^2\otimes \mathbb{C}^2
\end{equation*}
when $\ket{0}\bra{0}$ is measured at the second system.
\item Show the state vector $\ket{\psi}\in \mathbb{C}^2$ of the Bell state
when $\ket{+}\bra{+}$ is measured at the second system.
\item Show the state vector $\ket{\psi}\in \mathbb{C}^2$ of the Bell state
when $\ket{\varphi}\bra{\varphi}$ is measured at the second system where $\ket{\varphi}:=\alpha\ket{0}+\beta\ket{1}$.
%\item {[Advanced]} Show the reduced density matrix of system $B$ before the measurements in the quantum teleportation protocol.
\end{enumerate}
\end{frame}

\end{document}
